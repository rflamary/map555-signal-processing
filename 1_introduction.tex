
In this chapter we will introduce signal processing and discuss briefly the
numerous fundamental problems of signal processing. 

\section{Signal processing}
\label{sec:sigpro-intro}


\paragraph{Signal processing is everywhere}
Signal processing is a field that aim at modeling signals and providing automatic
processing of those signals. 
It has been heavily researched for several decades
and signal processing methods are central part of numerous technologies in
telecommunications, multi-media processing, compression and storage. 
In recent
years, tremendous results have been obtained by using modern machine learning
and artificial intelligence techniques.

\paragraph{Objective of this course} The objective of this course is to provide
an introduction to the very large field of signal processing. 
One fascinating
aspect of signal processing is that it is at the crossroad between Physics (to
generate the signals), Electronics (to measure the signals), Mathematics (to
model the signals) and Computer Science (to process the signals). In this sense,
Signal processing is a perfect example of a
multi-disciplinary field and a lot thee existing methods are known with other
names in other fields. An effort will be made to provide vocabulary coming from
the signal processing community but also statistics, machine learning and
computer science.

We plan on
introducing in this documents both the mathematical models, the numerical
algorithms used for their processing and several examples of real life
applications. The implementation of the signal processing methods in Python will
also be discussed with example code and existing
toolboxes. Note that most of the methods are introduced very briefly, but we
will always provide detailed references for a more in-depth study.


\paragraph{Content of the document} The course begins with a short introduction
of signal processing containing a few definitions and problems formulations
followed by bibliographical notes. Chapter \ref{chap:fourier-analog} provides 
a presentation of Fourier analysis and analog filtering
with some applicative examples such as modulation and Fourier optics in
astronomy. Chapter \ref{chap:dsp} introduces signal sampling and digital signal filtering that has
become the de-facto standard in practical applications. It also presents the very
important Fast Fourier Transform (FFT) algorithm and discuss some examples of
filtering in image processing. Chapter \ref{chap:random} discuss the random/stochastic aspects of signals and their optimal linear
filtering when modeled as as stochastic processes. The
modeling of speech is taken as an example for the study of
auto-regressive models. Chapter \ref{chap:sig_representations} briefly
introduces  several
signal representations commonly used such as the Discrete Cosine Transform
(DCT), and wavelet transforms used in
JPEG encoding and image reconstruction. The short time Fourier transform will also be introduced to model
non-stationary signals. Finally some recent approaches based on machine learning
such as dictionary learning and deep learning signal reconstruction will be
presented.


\section{Bibliographical notes}

This document was strongly inspired by a number of outstanding references books that
have been published over the years. In this section we discuss a few of those
strongly recommended references.  Suggestions to the author are
welcome to provide a curated list of "awesome" references for signal processing
similar to the lists available on GitHub\index{Reference books}. 

\paragraph{Signal processing}

\begin{itemize}
    
    \item Signals and Systems \cite{haykin2007signals}.
    \item Signals and Systems \cite{oppenheim1997signals}.
    \item Signal Analysis \cite{papoulis1977signal}.
    \item[] \hspace{1cm}
    
    \item Polycopiés from Stéphane Mallat and Éric Moulines \cite{mallat2015traitement}.
    \item Théorie du signal \cite{jutten2018theorie}.   

\end{itemize}

\paragraph{Analog signal processing and Fourier Transform}

\begin{itemize}
    \item Fourier Analysis and its applications \cite{vretblad2003fourier}
    \item Distributions et Transformation de Fourier \cite{roddier1985distributions}
  \end{itemize}

  \paragraph{Digital signal processing}

  \begin{itemize}
    \item   \url{https://www.numerical-tours.com/}
    
    \item  Discrete-time signal processing \cite{oppenheim1999discrete}.
  \end{itemize}

\paragraph{Random signals, stochastic processes}

\begin{itemize}
    \item Random variables and stochastic processes \cite{papoulis1965random}.
    \item \cite{ross1996stochastic}
    \item \cite{kay1993fundamentals}
\end{itemize}



\paragraph{Signal representations}
\begin{itemize}
    \item A Wavelet tour of signal processing \cite{mallat1999wavelet}.
    \item Wavelets and sub-band coding \cite{vetterli1995wavelets}.
\end{itemize}

\section{About this document}

\index{License}
This document contains lecture notes of MAP555 Signal Processing
Course from École Polytechnique. This document is currently being written and should be considered
unfinished and full of mistakes and typos. It should not be used yet as a
pedagogical support for a course. 

The document is available in \href{https://rflamary.github.io/map555-signal-processing/poly.pdf}{[PDF
format]} and \href{https://rflamary.github.io/map555-signal-processing/}{[HTML
format]} compiled automatically when the source is modified in the GitHub repository.

This work is licensed under a
\href{http://creativecommons.org/licenses/by-nc-sa/4.0/}{Creative Commons
Attribution-NonCommercial-ShareAlike 4.0 International License}. Reader are
encouraged to report typos and mistakes in the mathematical formulas and
proposed correction as Pull Requests on Github.


