\documentclass[a4paper, 10pt, dvipsnames]{book}
\usepackage[utf8]{inputenc}

\usepackage[mathjax,HTMLFilename={node-},latexmk,HomeHTMLFilename=index,]{lwarp}
\CSSFilename{map555.css}


\def\TheTitle{MAP555 : Signal Processing}

\boolfalse{FileSectionNames}  % If false, numbers the files.
\setcounter{tocdepth}{2}        % Include subsections in the \TOC.
\setcounter{secnumdepth}{3}     % Number down to subsections.
\setcounter{FileDepth}{1}       % Split \HTML\ files at sections
\booltrue{CombineHigherDepths}  % Combine parts/chapters/sections
\setcounter{SideTOCDepth}{1}    % Include subsections in the side\TOC
\HTMLTitle{MAP555 : Signal Processing}       % Overrides \title for the web page.
\HTMLAuthor{Rémi Flamary}        % Sets the HTML meta author tag.
\HTMLLanguage{en-US}            % Sets the HTML meta language.
\HTMLDescription{Lecture notes for MAP555 : Signal Processing}% Sets the HTML meta 
\MathJaxFilename{map555_mathjax.txt}
\HTMLPageBottom{<p><img alt="Creative
Commons License" style="border-width:0"
src="https://i.creativecommons.org/l/by-nc-sa/4.0/80x15.png" /> Rémi Flamary</p>}


\renewcommand{\theHTMLTitleSection}{\theHTMLTitle}

\usepackage{amsmath,amssymb,amsthm}       
\usepackage{lmodern}
%\usepackage{natbib}
%\usepackage[french]{babel}
\usepackage[dvipsnames]{xcolor}
\usepackage[pdftex,linktocpage,pdfstartview=FitH,colorlinks=true,linkcolor=blue,citecolor=magenta]{hyperref}
%\usepackage[pagebackref,hyperindex=true]{hyperref}
\usepackage{lipsum} 
% minitoc
\usepackage{minitoc}
\setcounter{minitocdepth}{2}
\mtcindent=10pt

\mtcsetfeature{minitoc}{open}{\vspace{1.5mm}}
\mtcsetfeature{minitoc}{close}{\vspace{1.5mm}}

\setcounter{tocdepth}{2}

\let\minitocORIG\minitoc
\renewcommand{\minitoc}{\minitocORIG \vspace{1.5em}}

%nouvelles polices pour minitoc
\renewcommand{\mtcfont}{\sffamily\small}
\renewcommand{\mtcSfont}{\sffamily\small\upshape\bfseries}
\renewcommand{\mtcSSfont}{\sffamily\small}
\renewcommand{\mtcSSSfont}{\sffamily\small}
\renewcommand{\mtifont}{\sffamily\large\bfseries}
\renewcommand{\ptifont}{\sffamily\Huge\bfseries}

\author{Rémi Flamary}

 
\begin{document} 

\title{MAP555 : Signal Processing \thanks{\textbf{Warning} : This document is currently being written and should be considered unfinished and full of mistakes and typos. It should not be used yet as a pedagogical support for a course.}}


\maketitle



\warpHTMLonly{ This document contains lecture notes from the Course MAP555 :
Signal Processing from the Applied Mathematics department of
\href{https://www.polytechnique.edu/en}{École
Polytechnique}.

The document is also in PDF format \href{poly.pdf}{here}

<a rel="license" href="http://creativecommons.org/licenses/by-nc-sa/4.0/"><img alt="Creative Commons License" style="border-width:0" src="https://i.creativecommons.org/l/by-nc-sa/4.0/88x31.png" /></a><br />This work is licensed under a <a rel="license" href="http://creativecommons.org/licenses/by-nc-sa/4.0/">Creative Commons Attribution-NonCommercial-ShareAlike 4.0 International License</a>.
}


\tableofcontents

%\part{Analog signal processing}
\chapter{Introduction}

\section{Signal processing}
\label{sec:sigpro-intro}

See Chap \ref{chap:fourier-analog} for intro to Fourier

\section{Definitions and signal properties}

\section{Bibliographical notes}

\cite{haykin2007signals,oppenheim1997signals}

\chapter{Fourier analysis and analog filtering}
\label{chap:fourier-analog}


%\lipsum[2-4]
\section{Fourier transform}
\label{sec:fourier-transform}
%\lipsum[2-4]
\section{Frequency response and filtering}
\label{sec:freq-response}
%\lipsum[2-4]
\section{Applications of analog signal processing}
\label{sec:appli-ft}
%\lipsum[2-4]

\chapter{Digital signal processing}
\label{chap:dsp}


\section{Sampling and Analog/Digital conversion}
\label{sec:}

\section{Digital filtering}
\label{sec:}

\section{Finite signals}
\label{sec:}

\section{Applications of DSP}
\label{sec:}


\chapter{Random signals}

\section{Random Signals and Correlations}
\label{sec:}

\section{Frequency representation of random signals}
\label{sec:}

\section{AR modeling and linear prediction}
\label{sec:}

\chapter{Signal representations}

\section{Short Time Fourier Transform}
\label{sec:}

\section{Common signal representations}
\label{sec:}

\section{Source separation and dictionary learning}
\label{sec:}


\section{Machine learning for signal processing}
\label{sec:}


\ForceHTMLTOC 
\ForceHTMLPage
\bibliographystyle{apalike}
\bibliography{biblio}

\end{document}