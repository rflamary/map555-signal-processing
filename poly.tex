\documentclass[a4paper, 10pt, dvipsnames]{book}
\usepackage[utf8]{inputenc}


\usepackage[mathjax,HTMLFilename={node-},latexmk,HomeHTMLFilename=index,]{lwarp}
\CSSFilename{map555.css}


\def\TheTitle{MAP555 : Signal Processing}

\boolfalse{FileSectionNames}  % If false, numbers the files.
\setcounter{tocdepth}{2}        % Include subsections in the \TOC.
\setcounter{secnumdepth}{3}     % Number down to subsections.
\setcounter{FileDepth}{1}       % Split \HTML\ files at sections
\booltrue{CombineHigherDepths}  % Combine parts/chapters/sections
\setcounter{SideTOCDepth}{1}    % Include subsections in the side\TOC
\HTMLTitle{MAP555 : Signal Processing}       % Overrides \title for the web page.
\HTMLAuthor{Rémi Flamary}        % Sets the HTML meta author tag.
\HTMLLanguage{en-US}            % Sets the HTML meta language.
\HTMLDescription{Lecture notes for MAP555 : Signal Processing}% Sets the HTML meta 
\MathJaxFilename{MAP555_mathjax.txt}
\HTMLPageBottom{<p><img alt="Creative
Commons License" style="border-width:0"
src="https://i.creativecommons.org/l/by-nc-sa/4.0/80x15.png" /> Rémi Flamary</p>}


\renewcommand{\theHTMLTitleSection}{\theHTMLTitle}

\usepackage{amsmath,amssymb,amsthm}       
\usepackage{lmodern}
%\usepackage[french]{babel}
\usepackage[dvipsnames]{xcolor}
\usepackage[pdftex,linktocpage,pdfstartview=FitH,colorlinks=true,linkcolor=blue,citecolor=magenta]{hyperref}
%\usepackage[pagebackref,hyperindex=true]{hyperref}
\usepackage{lipsum} 
% minitoc
\usepackage{minitoc}
\setcounter{minitocdepth}{2}
\mtcindent=10pt

\mtcsetfeature{minitoc}{open}{\vspace{1.5mm}}
\mtcsetfeature{minitoc}{close}{\vspace{1.5mm}}

\setcounter{tocdepth}{2}

\let\minitocORIG\minitoc
\renewcommand{\minitoc}{\minitocORIG \vspace{1.5em}}

%nouvelles polices pour minitoc
\renewcommand{\mtcfont}{\sffamily\small}
\renewcommand{\mtcSfont}{\sffamily\small\upshape\bfseries}
\renewcommand{\mtcSSfont}{\sffamily\small}
\renewcommand{\mtcSSSfont}{\sffamily\small}
\renewcommand{\mtifont}{\sffamily\large\bfseries}
\renewcommand{\ptifont}{\sffamily\Huge\bfseries}


\author{Rémi Flamary}
%\date{''date de fin de rédaction''} 
 
\begin{document} 

%%\pagenumbering{Alph} % to stop hyperref warnings
% \begin{titlepage}
%   \thispagestyle{empty}
%   \vspace*{-2cm}
%   %\hfill\includegraphics[width=4cm,alt={Course logo.}]{logo.png}
%   \vspace*{2cm}\vfill
%   {\noindent\huge\sffamily \TheTitle\\
%   \warpprintonly{\rule{\textwidth}{1pt}}\par}
%   \warpHTMLonly{<hr class="titlehrule">}
%   \vspace{\baselineskip} {\noindent\Large\sffamily Rémi Flamary}\par
%   \vspace{\baselineskip}
% %   \begin{center}\minipagefullwidth
% %     \begin{minipage}[t]{.9\linewidth}
% %       \noindent\LARGE\sffamily Corrections by:\\
% %       \begin{multicols}{2}
% %         \TheHeroes
% %       \end{multicols}
% %     \end{minipage}
% %   \end{center}
%   \vfill\vfill
% \end{titlepage} 
%\pagenumbering{roman}
\title{MAP555 : Signal Processing}
\maketitle

\warpHTMLonly{ This document contains lecture notes from the Course MAP555 :
Signal Processing from the Applied Mathematics department of
\href{https://www.polytechnique.edu/en}{École
Polytechnique}.

The document is also in PDF format \href{poly.pdf}{here}

<a rel="license" href="http://creativecommons.org/licenses/by-nc-sa/4.0/"><img alt="Creative Commons License" style="border-width:0" src="https://i.creativecommons.org/l/by-nc-sa/4.0/88x31.png" /></a><br />This work is licensed under a <a rel="license" href="http://creativecommons.org/licenses/by-nc-sa/4.0/">Creative Commons Attribution-NonCommercial-ShareAlike 4.0 International License</a>.
}


\tableofcontents

%\part{Analog signal processing}
\chapter{Introduction}

\section{Signal processing}
\label{sec:}

\lipsum[2-4]

\section{The reign of digital}

\section{Signals and definitions}


\chapter{Fourier analysis and analog filtering}


\label{sec:}
%\lipsum[2-4]
\section{Fourier transform}
\label{sec:}
%\lipsum[2-4]
\section{Frequency response and filtering}
\label{sec:}
%\lipsum[2-4]
\section{Applications of analog signal processing}
\label{sec:}
%\lipsum[2-4]

\chapter{Digital signal processing}

\chapter{Signal representations}

\end{document}